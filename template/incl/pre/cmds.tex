% cmds.tex : custom commands and environments
% ------------------------------------------------------------------------------
% This file contains definitions for custom commands and environments, used for
% shorthand notation for macros used often in the project


% Mathematical symbols ---------------------------------------------------------
\newcommand{\N}{\mathbb{N}}          % natural numbers
\newcommand{\Z}{\mathbb{Z}}          % integers
\newcommand{\Q}{\mathbb{Q}}          % rational numbers
\newcommand{\R}{\mathbb{R}}          % real numbers
\newcommand{\C}{\mathbb{C}}          % complex numbers
\newcommand{\1}{\mathbbm{1}}         % indicator function
\newcommand{\bigO}{\mathcal{O}}      % big O
\renewcommand{\vec}[1]{\bm{#1}}      % bold vector style

% Theorem enviroments ----------------------------------------------------------
% http://www.ctex.org/documents/packages/math/amsthdoc.pdf
\theoremstyle{plain}                 % Bold title, italic body text
\newtheorem{thm}{Theorem}[chapter]   % Theorems, numbered by chapter
\newtheorem{lem}[thm]{Lemma}         % Lemmas, numbered like theorems
\newtheorem{prop}[thm]{Proposition}  % Propositions, numbered like theorems
\newtheorem*{cor}{Corollary}         % Corollaries, unnumbered

\theoremstyle{definition}            % Bold title, upright body text
\newtheorem{defn}[thm]{Definition}   % Definitions, numbered like theorems
\newtheorem{exmp}[thm]{Example}      % Examples, numbered like theorems

% Figure commands --------------------------------------------------------------

% imgfig ("image figure")
% Shortcut command for inserting an image from the fig/img folder
% Arguments:
%   * (optional) figure width; percentage of text width (default: 0.75)
%   * file name (without fig/img/ AND without file extension); also used for the
% figure label as fig:name
%   * the figure caption
% Examples:
%   \imgfig{image-name}{Caption goes here}
%   \imgfig[0.5]{image-name}{Caption goes here}
\newcommand{\imgfig}[3][0.75]{
  \begin{figure}[htbp]
    \centering
    \includegraphics[width=#1\textwidth]{fig/img/#2}
    \caption{#3}
    \label{fig:#2}
  \end{figure}
}

% dimgfig ("double image figure")
% Shortcut command for inserting two images side by side
% Arguments:
%   * (optional) split percentage; how much of the text width is allocated to the
% left figure, the rest goes to the right (default: 0.5, i.e. even split)
%   * file name for the left figure, without fig/img/ and file extension
%   * caption for the left figure
%   * file name for the right figure, without fig/img/ and file extension
%   * caption for the right figure
% Examples:
%   \dimgfig{img1}{First caption}{img2}{Second caption}
%   \dimgfig[0.3]{img1}{First caption}{img2}{Second caption}
% Alteratively, see
% https://en.wikibooks.org/wiki/LaTeX/Floats,_Figures_and_Captions#Subfloats
\newcommand{\dimgfig}[5][0.5]{
  \ifx\dimgleftwidth\undefined
    \newlength{\dimgleftwidth}
    \newlength{\dimgrightwidth}
  \fi
  \setlength{\dimgleftwidth}{#1\textwidth-0.02\textwidth}
  \setlength{\dimgrightwidth}{0.96\textwidth-\dimgleftwidth}
  \begin{figure}[htbp]
    \centering
    \begin{minipage}[t]{\dimgleftwidth}
      \centering
      \includegraphics[width=\linewidth]{fig/img/#2}
      \caption{#3}
      \label{fig:#2}
    \end{minipage}
    \hfill
    \begin{minipage}[t]{\dimgrightwidth}
      \centering
      \includegraphics[width=\linewidth]{fig/img/#4}
      \caption{#5}
      \label{fig:#4}
    \end{minipage}
  \end{figure}
}
