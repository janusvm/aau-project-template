% LaTeX skabelon, master fil
% ------------------------------------------------------------------------------
% Dette er hovedfilen i projektet, hvori indholdet fra alle input-filer samles;
% tekst, billeder, literaturhenvisninger, osv.


% LaTeX' 'book' class er den documentclass, der har de mest fleksible muligheder
% for at projekt som dette. Se https://tex.stackexchange.com/a/36989/118167
%
% De ekstra argumenter i firkantklammerne sætter grundskriftstørrelsen til 11pt,
% sidestørrelse til A4, layout til to-sidet, og gør så nye chapters starter på
% en højreside.
\documentclass[11pt,a4paper,twoside,openright]{book}

% Variable til automatisk at sætte titler, forfattere, osv. diverse steder
% (såsom forsiden og titelbladet). For flere muligheder skal man justere de
% steder, de bliver brugt, manuelt.
\def \projecttitle       {\LaTeX{} Skabelon}
\def \projectsubtitle    {Kom Godt I Gang Med Projektskrivning}
\def \projectdegree      {Matematik-\O{}konomi}
\def \projectsemester    {1}
\def \projectsupervisors {Janus S. Valberg-Madsen}
\def \projectgroup       {Gruppe XYZ}
\def \projectauthors     {
  Nova B. Valberg-Madsen\\
  Line B. Olsen
}

% Den medfølgende 'macros' fil definerer kommandoer, der genererer en titelside
% i AAU-stil.  Bemærk, at kommandoer normalt puttes ind i preamblet, men
% 'macros' er skrevet af en anden person, så derfor har jeg beholdt den som
% seperat fil.
\input{incl/misc/macros}

% Preamblet indeholder alle de ting, som skal angives før selve indholdet af
% dokumentet starter. Kommandoen \input svarer til at kopiere indholdet fra
% filen ind, der hvor kommandoen kaldes.
% LaTeX template, preamble file
% ------------------------------------------------------------------------------
% This is the preamble file for the project, in which package dependencies and
% custom commands should be defined.


% Base packages ----------------------------------------------------------------

\usepackage[T1]{fontenc}     % adds support for printing accented characters
\usepackage[utf8]{inputenc}  % adds support for inputting accented characters
\usepackage[english]{babel}  % set document language
\usepackage[square]{natbib}  % set citation style to use square brackets
\usepackage{hyperref}        % insert clickable hyperlinks
\usepackage{lastpage}        % reference last page (used to count numbered pages)


% Math packages ----------------------------------------------------------------

\usepackage{mathtools}  % math commands and environments
\usepackage{amssymb}    % additional math symbols
\usepackage{amsthm}     % theorem environments
\usepackage{bbm}        % improved 'blackboard bold' styling


% Figure packages --------------------------------------------------------------

\usepackage[dvipsnames]{xcolor}    % font colouring functionality
\usepackage{graphicx}              % handling of images
\usepackage{booktabs}              % improved tables
\usepackage{tikz}                  % drawing vector graphics
\usepackage{algorithm}             % algorithm floats
\usepackage[noend]{algpseudocode}  % pseudocode listings
\usepackage{listings}              % source code listings


% Styling packages -------------------------------------------------------------

\usepackage{microtype}             % improve full justification and font kerning
\usepackage{emptypage}             % suppress page numbering on empty pages
\usepackage{fancyhdr}              % easy customisation of headers and footers
\usepackage{titlesec}              % easy customisation of section titles
\usepackage[margin=3cm]{geometry}  % adjust page margins
\usepackage[toc,page]{appendix}    % improved appendix control


% Font packages ----------------------------------------------------------------
% With pdflatex, fonts are limited to those provided in packages (as opposed to
% xelatex or lualatex, which can use system fonts).  The typeface is changed by
% simply importing the package for it.
%
% See
% * http://www.tug.org/pracjourn/2006-1/hartke/hartke.pdf
% * http://www.tug.dk/FontCatalogue/

\usepackage{lmodern}  % improved default font


% Customisations ---------------------------------------------------------------

% Theorem enviroments
\theoremstyle{plain}
\newtheorem{thm}{Theorem}[chapter]
\newtheorem{lem}[thm]{Lemma}
\newtheorem{prop}[thm]{Proposition}
\newtheorem*{cor}{Corollary}

\theoremstyle{definition}
\newtheorem{defn}[thm]{Definition}
\newtheorem{exmp}[thm]{Example}

% Sets
\newcommand{\N}{\mathbb{N}}  % natural numbers
\newcommand{\Z}{\mathbb{Z}}  % integers
\newcommand{\Q}{\mathbb{Q}}  % rational numbers
\newcommand{\R}{\mathbb{R}}  % real numbers
\newcommand{\C}{\mathbb{C}}  % complex numbers

% Other operators and concepts
\newcommand{\1}{\mathbbm{1}}     % indicator function
\newcommand{\bigO}{\mathcal{O}}  % big O

% Headers and footers
\fancyhead{}                      % clear default header fields
\fancyhead[LE,RO]{\projectgroup}  % left on even, right on odd
\fancyhead[RE]{\leftmark}         % chapter name
\fancyhead[LO]{\rightmark}        % sections name
\fancyfoot{}                      % clear default footer fields
\fancyfoot[CE,CO]{\thepage}       % centered page numbers
\pagestyle{fancy}                 % activate fancy style

% Chapter titles
% http://mirrors.dotsrc.org/ctan/macros/latex/contrib/titlesec/titlesec.pdf
\titleformat
{\chapter}
[hang]
{\Huge\bfseries}
{\thechapter\enspace\textcolor{BrickRed}{|}\enspace}
{0pt}
{\Huge\bfseries}

% Predefined TikZ node styles
\tikzstyle{point} = [fill,shape=circle,minimum size=3pt,inner sep=0pt]
\tikzstyle{edge} = [fill=white,midway,inner sep=1pt]

% Listing style
\lstdefinestyle{custompy}{
  belowcaptionskip=\baselineskip,
  breaklines=true,
  language=Python,
  showstringspaces=false,
  basicstyle=\small\ttfamily,
  keywordstyle=\bfseries\color{green!40!black},
  commentstyle=\itshape\color{purple!40!black},
  identifierstyle=\color{blue},
  stringstyle=\color{orange},
}
\lstset{language=Python,style=custompy,captionpos=b}

% Alt tekst skal inputtes i et 'document' environment.
\begin{document}

% \frontmatter angiver, at det efterfølgende indhold ikke skal tælles med som
% nummererede sider, men istedet have deres egne sidetal (i romertal som
% standard).
\frontmatter
<<#header>>
% incl/misc/frontpage.tex : <<&title>>
% ------------------------------------------------------------------------------
<</header>>


\backgroundsetup{
  scale = 1,
  angle=0,
  opacity=1,
  contents = {
    \includegraphics[width=\paperwidth,height=\paperheight]{fig/img/aau/waves.pdf}
  }
}
\BgThispage
<<#pdfbookmark>>
\pdfbookmark[0]{<<&name>>}{<<&label>>}
<</pdfbookmark>>
\begin{titlepage}
  \centering
  \phantom{}
  \vspace{2cm}

  % <<&badge>>
  \begin{minipage}[c]{0.2\paperwidth}
    \centering
    \makebox[0pt]{
      <<#header>>
% fig/tikz/aau-badge.tex : <<&title>>
% ------------------------------------------------------------------------------
<</header>>

\begin{tikzpicture}
  % <<&figdesc>>
  \node[circle,color=white,fill=white,minimum size=1.175\textwidth] at (0,0) {};
  \node at (0,0) {\includegraphics[width=\textwidth]{fig/img/aau/logo-circle.pdf}};
\end{tikzpicture}

    }
  \end{minipage}

<<#main>>
  % <<&desc>>
  \vspace{4cm}
  {\fontfamily{bch}\selectfont
    \fboxsep0pt\colorbox{white}{
      \begin{minipage}{\textwidth}
        \centering
        \color{AAUblue1}

        \vspace{2em}
        {\Huge\bfseries\projecttitle}

        {\Large\bfseries\projectsubtitle}

        \bigskip
        \parbox{\textwidth}{\centering\large\projectauthors}

        \bigskip
        {\bfseries\large{\projectnumber}<<&project>>, <<&group>> \projectgroup, \projectdegree}
        \vspace{2em}
      \end{minipage}
    }
  }
<</main>>

\end{titlepage}

% LaTeX template, example title page
% ------------------------------------------------------------------------------
% The title page is generated by the command \aautitlepage, which is defined in
% /incl/pre/ext/aautitlepage.sty


\pdfbookmark[0]{Title page}{titlepage}
\aautitlepage{
  \englishprojectinfo{
    \projecttitle
  }{
    \projecttheme
  }{
    \projectperiod
  }{
    Group \projectgroup
  }{
    \parbox[t]{\textwidth}{\projectauthors}
  }{
    \parbox[t]{\textwidth}{\projectsupervisors}
  }{
    \today
  }
}{
  \textbf{Dept. of Mathematical Sciences}\\
  Skjernvej 4A\\
  DK-9220 Aalborg Ø\\
  \href{http://math.aau.dk}{http://math.aau.dk}
}{
  <<#header>>
% incl/misc/abstract.tex : <<&title>>
% ------------------------------------------------------------------------------
<<#description>>
% <<&line>>
<</description>>
<</header>>

}


% Automatisk indholdsfortegnelse baseret på overskrifterne i inputfilerne.
\tableofcontents

% \mainmatter angiver hovedindholdet, som nummereres med arabiske tal.
\mainmatter

% Inputfiler bør opdeles, således at hver fil indeholdet ét kapitel.
% Kommandoen \include indsætter en ny side efterfulgt af indholdet i inputfilen.
% LaTeX template, example input file
% --------------------------------------------------------------------------------------------------

\chapter{First Example}
\label{ch:first-example}

This is an example chapter with content.
Aside from \verb!\chapter!, there are several levels available to use for partitioning your body text:
%
\begin{itemize}
\item \texttt{part}
\item \texttt{chapter}
\item \texttt{section}
\item \texttt{subsection}
\item \texttt{subsubsection}
\item \texttt{paragraph}
\item \texttt{subparagraph}
\end{itemize}
%
Each level is a subsection of the above level.
Titles are added automatically to the table of contents.
See more at \url{https://en.wikibooks.org/wiki/LaTeX/Document_Structure#Sectioning_commands}.


% --------------------------------------------------------------------------------------------------
\section{Environments}
\label{sec:sec-within-chap}

In \LaTeX{}, you are going to be using many different kinds of \emph{environments}.
These are scopes denoted with \verb!\begin{...}! and \verb!\end{...}!, enclosing special content such as lists, figures, equations, etc.
Table \ref{tab:my-table} lists some commonly used environments.

\begin{table}[htbp]
  \centering
  \begin{tabular}{ll}
    \toprule
    \textbf{Environment} & \textbf{Function} \\
    \midrule
    \texttt{document} & Document contents \\
    \texttt{table} & Floating table such as this one \\
    \texttt{figure} & Floating figure \\
    \texttt{equation} & Numbered equation \\
    \texttt{align} & Aligned, multiple equations \\
    \texttt{itemize} & Bulleted list \\
    \texttt{enumerate} & Numbered list \\
    \texttt{description} & Descriptive list \\
    \bottomrule
  \end{tabular}
  \caption{Common \LaTeX{} environments and their function}
  \label{tab:my-table}
\end{table}


\subsection{Lists}
\label{sec:lists}

There are three essential list structures: \texttt{itemize}, \texttt{enumerate}, and \texttt{description}.
The \texttt{itemize} variant produces a simple bullet list.
Each item in the list are prepended by the \verb!\item! command.
%
\begin{verbatim}
  \begin{itemize}
    \item First item
    \item Second item
    \item Third item
  \end{itemize}
\end{verbatim}
%
\begin{itemize}
\item First item
\item Second item
\item Third item
\end{itemize}
%
The \texttt{enumerate} variant uses the same syntax for items as \texttt{itemize}, but produces a numbered list.
%
\begin{verbatim}
  \begin{enumerate}
    \item First item
    \item Second item
    \item Third item
  \end{enumerate}
\end{verbatim}
%
\begin{enumerate}
\item First item
\item Second item
\item Third item
\end{enumerate}
%
Finally, the \texttt{description} list in which \verb!\item! is given an item name as an optional argument, and the contents of the line is a description of that item.
This produces a list where the item names are typeset in bold followed by their descriptions as normal text.
%
\begin{verbatim}
  \begin{description}
    \item[First item] Description of first item
    \item[Second item] Description of second item
    \item[Third item] Description of third item
  \end{description}
\end{verbatim}
%
\begin{description}
\item[First item] Description of first item
\item[Second item] Description of second item
\item[Third item] Description of third item
\end{description}


\subsection{Equations}
\label{sec:equations}

One of the main reasons why people use \LaTeX{} is the beautiful math typesetting.
There are several different math environments to suit your needs, and most come in a numbered and unnumbered variants.
For example, the code
%
\begin{verbatim}
\begin{equation}
  \label{eq:1}
  e^{i\pi} - 1 = 0
\end{equation}
\end{verbatim}
%
produces the ouput
%
\begin{equation}
  \label{eq:1}
  e^{i\pi} + 1 = 0 ,
\end{equation}
%
and since it was given a label, it can be referenced with the command \verb!\eqref{eq:1}!, which produces a clickable reference in parentheses, \eqref{eq:1}.
If instead of \texttt{equation} you put \texttt{equation*}, the equation does not get a number.
Equivalently, you can use \verb!\[...\]!, so the code \verb!\[ e^{i\pi} + 1 = 0 \]! produces
\[ e^{i\pi} + 1 = 0 .\]

If you need multiple, aligned equations, e.g. for step-by-step calculations, use the \texttt{align} environment, which aligns the contents at \texttt{\&} characters.
For example,
%
\begin{verbatim}
\begin{align*}
  (x + y)^{2} &= x^{2} + xy + yx + y^{2} \\
              &= x^{2} + y^{2} + 2xy
\end{align*}
\end{verbatim}
%
produces
%
\begin{align*}
  (x + y)^{2} &= x^{2} + xy + yx + y^{2} \\
              &= x^{2} + y^{2} + 2xy .
\end{align*}
%
The double backslash denotes a line break.
Note the asterisk; like with \texttt{equation}, \texttt{align} has both a numbered and unnumbered version.
The numbered version has a seperate number for each line.

For rendering inline math, e.g. \(\cos^{2}\theta + \sin^{2}\theta = 1\), use \verb!\(...\)!.
Alternatively, you can also use \verb!$...$!, but \verb!\(...\)! has improved spacing and error messages.

See \url{https://en.wikibooks.org/wiki/LaTeX/Mathematics} for a good reference of symbols and commands.


\subsection{Floats}
\label{sec:floats}

Figures in \LaTeX{} are input as so-called \emph{floats} using the \texttt{figure} environment.
A floating object cannot be broken over a page, so the figure will be repositioned depending on the available space on the page.
The syntax is as follows:
%
\begin{verbatim}
\begin{figure}[placement]
  \centering
  \includegraphics[options]{path/to/image}
  \caption{The figure caption}
  \label{fig:label}
\end{figure}
\end{verbatim}
%
The optional argument \texttt{placement} can be either of \texttt{h} (here), \texttt{t} (top of page), \texttt{b} (bottom of page), or \texttt{p} (put on special page with only floats).
The \verb!\centering! command is there to center the image.
Among the options available for \verb!\includegraphics!, the most important one for you will probably be \texttt{width}.
To make the image take up half the page (within margins), use \verb!width=0.5\textwidth!.
All figures should have a caption, which is set with the \verb!\caption! command, and the \verb!\label! lets us reference it (for example, \verb!Figure \ref{fig:me}! becomes Figure \ref{fig:me}).

% LaTeX template, example figure input file
% ------------------------------------------------------------------------------
% This file contains all the code involved in inputting a single image figure.
% While this code can just as well be inserted directly into the documents with
% the body text, I like to keep my projects as modular as possible.


\begin{figure}[htbp]
  \centering
  \includegraphics[width=0.5\textwidth]{fig/img/me.jpg}
  \caption{A picture of me responding to emails from my students}
  \label{fig:me}
\end{figure}


For more info, see \url{https://en.wikibooks.org/wiki/LaTeX/Floats,_Figures_and_Captions}.

% LaTeX template, example input file
% ------------------------------------------------------------------------------

\chapter{Andet Eksempel}
\label{ch:second-example}

Her er endnu et eksempel på en inputfil.


\section{Brugerdefinerede Environments og Kommandoer}
\label{sec:custom}

Selvom \LaTeX{} har mange nyttige kommandoer indbygget, så vil I ofte have brug for at definere jeres egne.
I denne skabelon har jeg inkluderet nogle eksempler på brugerdefinerede environments og kommandoer i preamblet (\texttt{preamble.tex}).

Sådanne kommandoer kan spare jer for at skulle gentage jer selv for meget, når I skriver lange dokumenter såsom et semesterprojekt.
For eksempel kan man, i stedet for at skrive \verb!\mathbb{N}! hver gang, man skal bruge symbolet for de naturlige tal, definere en kortere kommando som f.eks. \verb!\N!.
Syntaksen er som følger:
% 
\begin{verbatim}
\newcommand{name}[num]{definition}
\end{verbatim}
%
hvor \texttt{name} er kommandoens navn, f.eks. \verb!\N!, \texttt{num} er antallet af argumenter, kommandoen tager (udelad firkantklammerne, hvis kommandoen ingen argumenter tager), og \texttt{definition} er outputtet af kommandoen, f.eks. \verb!\mathbb{N}!.


\subsection{Definition, Sætning, Bevis}
\label{sec:thms}

I et matematikprojekt får man brug for at skrive matematiske definitioner, propositioner, lemmaer, sætninger, osv.
Pakken \texttt{amsthm} giver en simpel måde at definere environments til formålet:
%
\begin{verbatim}
\newtheorem{name}{Printed output}[numberby]
\end{verbatim}
%
Et par eksempler er inkluderet i preamblet.
Se \url{https://en.wikibooks.org/wiki/LaTeX/Theorems} for mere information.
%
\begin{defn}[Rationelle tal]
  A reelt tal \(r\) kaldes \emph{rationelt}, hvis der eksisterer heltal \(p\) og \(q\) med \(q \neq 0\) sådan at \(r = p/q\)
  (et reelt tal, som ikke er rationelt kaldes \emph{irrationelt}).
  Mængden af rationelle tal skrives som \(\Q\).
\end{defn}
%
\begin{thm}[Eksempel på Navngivet Sætning]
  \(\sqrt{2}\) er irrationel.
\end{thm}
%
\begin{proof}
  Antag \(\sqrt{2} \in \Q\) og lad \(k = \min\{\, N \in \N : N\sqrt{2} \in \N \,\}\).
  Men så har vi, at \(k (\sqrt{2} - 1) = (k\sqrt{2} - k) \in \N\), og \(k(\sqrt{2} - 1) < k\), hvilket er en modstrid.
  Dermed har vi, at \(\sqrt{2} \notin \Q\).
\end{proof}


\section{Kildehenvisninger}
\label{sec:citations}

Når I bruger andre folks værker, så skal i henvise til dem.
Henvisninger i \LaTeX{} håndteres normal med BibTeX, et program, der identificerer ressourcer fra en bibliografi-fil, I skriver.
Literaturhenvisninger skal defineres i en \texttt{.bib}-fil og inkluderes med kommandoen \verb!\bibliography!.
Hver ressource i denne fil skal have en type og et ID, som bruges til at indsætte henvisninger til den.
Syntaksen for at indsætte en henvisning er 
%
\begin{verbatim}
\citep[scope]{bibkey}
\end{verbatim}
%
hvor \texttt{bibkey} er det ID, I giver en ressource i \texttt{.bib}-filen, og \texttt{scope} kan bruges til at henvise til speficikke afsnit eller sidetal.

Når I henviser til en ressource, bliver den automatisk tilføjet til literatturlisten, og i PDF-filen bliver henvisninger i teksten til klikbare links til denne liste.
For eksempel, ved at henvise til \citep[s. 104-110]{edpenn}, tilføjes denne til listen på sidste side.

Der er mange forskellige typer af ressourcer i BibTeX, og hver type har sit eget sæt af felter, som skal eller kan udfyldes.
For mere information, se \url{https://en.wikibooks.org/wiki/LaTeX/Bibliography_Management#BibTeX}.

% Appendicer
\begin{appendices}
  % LaTeX template, example appendix file
% -------------------------------------------------------------------------------------------------

\chapter{Pseudocode}
\label{app:pseudocode}

To include pseudocode in your project, use the \texttt{algorithmic} environment (place inside an \texttt{algorithm} environment for figure-like floating):
%
\begin{verbatim}
\begin{algorithm}
  \begin{algorithmic}
    % pseudocode here...
  \end{algorithmic}
  \caption{algorithm title}
  \label{alg:label}
\end{algorithm}
\end{verbatim}

\begin{algorithm}
  \begin{algorithmic}
    \Procedure{bubblesort}{\,\(a_{1}, \dotsc, a_{n}\) : real numbers with \(n \geq 2\)\,}
    \For{\(i \coloneqq 1\) to \(n - 1\)}
    \For{\(j \coloneqq 1\) to \(n - i\)}
    \If{\(a_{j} > a_{j+1}\)}
    interchange \(a_{j}\) and \(a_{j+1}\)
    \EndIf
    \EndFor
    \EndFor
    \EndProcedure
    \Comment{\(a_{1}, \dotsc, a_{n}\) are now in increasing order}
  \end{algorithmic}
  \caption{Bubble Sort}
  \label{alg:some}
\end{algorithm}

See \url{https://en.wikibooks.org/wiki/LaTeX/Algorithms#Typesetting_using_the_algorithmicx_package} for a quick reference of the syntax.
  % LaTeX skabelon, eksempel på appendiks-fil
% ------------------------------------------------------------------------------

\chapter{Python Kode}
\label{app:code}

Hvis I implementerer algoritmer i et specifikt programmeringssprog, vil I muligvis inkludere kildekoden i et appendiks som dette.
Pakken \texttt{listings} gør det muligt at inkludere kildekode med syntax highlighting for mange forskellige sprog.

Med environmentet \texttt{lstlisting} kan man indsætte kode direkte, men for det meste vil I typisk bruge \verb!\lstinputlisting! til at indsætte kode fra en fil, f.eks.
%
\begin{verbatim}
\lstinputlisting[options]{path/to/source_file}
\end{verbatim}
%
Nogle nyttige \texttt{options} for denne kommando er:
%
\begin{description}
\item[caption] sætter caption for koden
\item[label] sætter label for koden (til \verb!\ref!)
\item[language] sætter syntaksen til et bestemt sprog lokalt
\item[style] sætter highlighter stilen lokalt
\item[firstline] springer linjerne før den angivede linje over
\item[lastline] stopper med at læse linjer, når den angivede linje nåes
\end{description}
%
I preamblet kan man også sætte globale options med \verb!\lstset!.
For eksempel, hvis al ens kildekode er skrevet i Python, kan man skrive \verb!\lstset{language=Python}!, og så behøver man ikke bruge sætte \texttt{language} i hver enkel \verb!\lstinputlisting!.

\lstinputlisting[firstline=5,caption=Bubble Sort in Python,label=py:bubble]{py/bubblesort.py}

For mere information, se \url{https://en.wikibooks.org/wiki/LaTeX/Source_Code_Listings}.

\end{appendices}

% \backmatter er til ekstra ting og sager. Overskrifter er ikke nummererede.
\backmatter

% Automatiske lister af alle figurer og tabeller, baseret på deres captions.
\listoffigures
\listoftables

% Automatisk liste af referencer, baseret på, hvilke referencer fra bib-filen,
% der rent faktisk blev henvist til i dokumentet.
\bibliographystyle{apalike}
\bibliography{incl/bib/bibliography}

\end{document}
