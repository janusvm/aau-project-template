% LaTeX template, example input file
% ------------------------------------------------------------------------------

\chapter{Andet Eksempel}
\label{ch:second-example}

Her er endnu et eksempel på en inputfil.


\section{Brugerdefinerede Environments og Kommandoer}
\label{sec:custom}

Selvom \LaTeX{} har mange nyttige kommandoer indbygget, så vil I ofte have brug for at definere jeres egne.
I denne skabelon har jeg inkluderet nogle eksempler på brugerdefinerede environments og kommandoer i preamblet (\texttt{preamble.tex}).

Sådanne kommandoer kan spare jer for at skulle gentage jer selv for meget, når I skriver lange dokumenter såsom et semesterprojekt.
For eksempel kan man, i stedet for at skrive \verb!\mathbb{N}! hver gang, man skal bruge symbolet for de naturlige tal, definere en kortere kommando som f.eks. \verb!\N!.
Syntaksen er som følger:
% 
\begin{verbatim}
\newcommand{name}[num]{definition}
\end{verbatim}
%
hvor \texttt{name} er kommandoens navn, f.eks. \verb!\N!, \texttt{num} er antallet af argumenter, kommandoen tager (udelad firkantklammerne, hvis kommandoen ingen argumenter tager), og \texttt{definition} er outputtet af kommandoen, f.eks. \verb!\mathbb{N}!.


\subsection{Definition, Sætning, Bevis}
\label{sec:thms}

I et matematikprojekt får man brug for at skrive matematiske definitioner, propositioner, lemmaer, sætninger, osv.
Pakken \texttt{amsthm} giver en simpel måde at definere environments til formålet:
%
\begin{verbatim}
\newtheorem{name}{Printed output}[numberby]
\end{verbatim}
%
Et par eksempler er inkluderet i preamblet.
Se \url{https://en.wikibooks.org/wiki/LaTeX/Theorems} for mere information.
%
\begin{defn}[Rationelle tal]
  A reelt tal \(r\) kaldes \emph{rationelt}, hvis der eksisterer heltal \(p\) og \(q\) med \(q \neq 0\) sådan at \(r = p/q\)
  (et reelt tal, som ikke er rationelt kaldes \emph{irrationelt}).
  Mængden af rationelle tal skrives som \(\Q\).
\end{defn}
%
\begin{thm}[Eksempel på Navngivet Sætning]
  \(\sqrt{2}\) er irrationel.
\end{thm}
%
\begin{proof}
  Antag \(\sqrt{2} \in \Q\) og lad \(k = \min\{\, N \in \N : N\sqrt{2} \in \N \,\}\).
  Men så har vi, at \(k (\sqrt{2} - 1) = (k\sqrt{2} - k) \in \N\), og \(k(\sqrt{2} - 1) < k\), hvilket er en modstrid.
  Dermed har vi, at \(\sqrt{2} \notin \Q\).
\end{proof}


\section{Kildehenvisninger}
\label{sec:citations}

Når I bruger andre folks værker, så skal i henvise til dem.
Henvisninger i \LaTeX{} håndteres normal med BibTeX, et program, der identificerer ressourcer fra en bibliografi-fil, I skriver.
Literaturhenvisninger skal defineres i en \texttt{.bib}-fil og inkluderes med kommandoen \verb!\bibliography!.
Hver ressource i denne fil skal have en type og et ID, som bruges til at indsætte henvisninger til den.
Syntaksen for at indsætte en henvisning er 
%
\begin{verbatim}
\citep[scope]{bibkey}
\end{verbatim}
%
hvor \texttt{bibkey} er det ID, I giver en ressource i \texttt{.bib}-filen, og \texttt{scope} kan bruges til at henvise til speficikke afsnit eller sidetal.

Når I henviser til en ressource, bliver den automatisk tilføjet til literatturlisten, og i PDF-filen bliver henvisninger i teksten til klikbare links til denne liste.
For eksempel, ved at henvise til \citep[s. 104-110]{edpenn}, tilføjes denne til listen på sidste side.

Der er mange forskellige typer af ressourcer i BibTeX, og hver type har sit eget sæt af felter, som skal eller kan udfyldes.
For mere information, se \url{https://en.wikibooks.org/wiki/LaTeX/Bibliography_Management#BibTeX}.