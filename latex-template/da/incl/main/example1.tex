% LaTeX skabelon, eksempel på inputfil
% ------------------------------------------------------------------------------

\chapter{Første Eksempel}
\label{ch:first-example}

Dette er et eksempel på et kapital med indhold.
Udover \verb!\chapter! findes der adskillige kommandoer for at niveauinddele ens tekst:
%
\begin{itemize}
\item \texttt{part}
\item \texttt{chapter}
\item \texttt{section}
\item \texttt{subsection}
\item \texttt{subsubsection}
\item \texttt{paragraph}
\item \texttt{subparagraph}
\end{itemize}
%
Hvert niveau er en underafsnit af niveauet over det.
Overskrifter tilføjes automatisk til indholdsfortegnelsen.
Se mere på \url{https://en.wikibooks.org/wiki/LaTeX/Document_Structure#Sectioning_commands}.


% ------------------------------------------------------------------------------
\section{Environments}
\label{sec:sec-within-chap}

I \LaTeX{} vil I komme til at skulle bruge mange forskellige slags \texttt{environments}.
De åbnes med \verb!\begin{...}! og lukkes med \verb!\end{...}! og har specifikke formål, såsom at lave lister, figurer, ligninger, osv.
I Tabel \ref{tab:my-table} er nogle af de mest almindelige environments.

\begin{table}[htbp]
  \centering
  \begin{tabular}{ll}
    \toprule
    \textbf{Environment} & \textbf{Function} \\
    \midrule
    \texttt{document} & Document contents \\
    \texttt{table} & Floating table such as this one \\
    \texttt{figure} & Floating figure \\
    \texttt{equation} & Numbered equation \\
    \texttt{align} & Aligned, multiple equations \\
    \texttt{itemize} & Bulleted list \\
    \texttt{enumerate} & Numbered list \\
    \texttt{description} & Descriptive list \\
    \bottomrule
  \end{tabular}
  \caption{Common \LaTeX{} environments and their function}
  \label{tab:my-table}
\end{table}


\subsection{Lister}
\label{sec:lists}

Der er tre essentielle slags lister: \texttt{itemize}, \texttt{enumerate}, and \texttt{description}.
Varianten \texttt{itemize} producerer en simpel punktliste.
Hvert punkt angives med kommandoen \verb!\item!.
%
\begin{verbatim}
  \begin{itemize}
    \item Første punkt 
    \item Andet punkt
    \item Tredje punkt
  \end{itemize}
\end{verbatim}
%
\begin{itemize}
\item Første punkt 
\item Andet punkt
\item Tredje punkt
\end{itemize}
%
Varianten \texttt{enumerate} bruger samme syntaks som \texttt{itemize}, men producerer en nummereret liste.
%
\begin{verbatim}
  \begin{enumerate}
    \item Første punkt 
    \item Andet punkt
    \item Tredje punkt
  \end{enumerate}
\end{verbatim}
%
\begin{enumerate}
\item Første punkt 
\item Andet punkt
\item Tredje punkt
\end{enumerate}
%
I \texttt{description}-listen tager \verb!\item! kommandoerne navnet på det, som beskrives, som argument, og resten af linjen udgør selve beskrivelsen.
Dette producerer en liste, hvor navnene er skrevet med fed, og deres beskrivelser er normal tekst.
%
\begin{verbatim}
  \begin{description}
    \item[Første ting] Beskrivelse af den første ting
    \item[Anden ting] Beskrivelse af den anden ting
    \item[Tredje ting] Beskrivelse af den tredje ting
  \end{description}
\end{verbatim}
%
\begin{description}
\item[Første ting] Beskrivelse af den første ting
\item[Anden ting] Beskrivelse af den anden ting
\item[Tredje ting] Beskrivelse af den tredje ting
\end{description}


\subsection{Ligninger}
\label{sec:equations}

En af hovedårsagerne bag, hvorfor folk bruger \LaTeX{}, er hvor godt matematiske formler ser ud.
Der er mange forskellige matematikrelaterede environments til forskellige formål, og de fleste har både nummererede og unummererede varianter.
For eksempel, hvis man skriver
%
\begin{verbatim}
\begin{equation}
  \label{eq:1}
  e^{i\pi} - 1 = 0 ,
\end{equation}
\end{verbatim}
%
så får man følgende output:
%
\begin{equation}
  \label{eq:1}
  e^{i\pi} + 1 = 0 ,
\end{equation}
%
og fordi den har en label, så kan man referere til den med kommandoen \verb!\eqref{eq:1}!, hvilket producerer en klikbar reference i paranteser, \eqref{eq:1}.
Hvis man i stedet for \texttt{equation} bruger \texttt{equation*}, så får ligninger \emph{ikke} et nummer.
Alternativt kan man bruge \verb!\[...\]!, så hvis man skriver \verb!\[ e^{i\pi} + 1 = 0 \]!, så får man
\[ e^{i\pi} + 1 = 0 .\]

Hvis man skal have flere ligninger i træk, f.eks. til trin-for-trin udregninger, så brug \texttt{align}, som får ligninger til at justere sig omkring \texttt{\&} symboler.
For eksempel, hvis man skriver
%
\begin{verbatim}
\begin{align*}
  (x + y)^{2} &= x^{2} + xy + yx + y^{2} \\
              &= x^{2} + y^{2} + 2xy
\end{align*}
\end{verbatim}
%
så vil \(=\) i de to ligninger stå lige over hinanden;
%
\begin{align*}
  (x + y)^{2} &= x^{2} + xy + yx + y^{2} \\
              &= x^{2} + y^{2} + 2xy .
\end{align*}
%
Dobbelt backslash indsætter et linjeskift.
Bemærk stjernen; ligesom med \texttt{equation} så har \texttt{align} både en nummereret og unummereret udgave.
Den nummererede udgave har et seperat nummer for \emph{hver} linje.

For at skrive matematik i brødteksten, f.eks. \(\cos^{2}\theta + \sin^{2}\theta = 1\), brug \verb!\(...\)!.
Alternativt kan man bruge \verb!$...$!, men \verb!\(...\)! har bedre mellemrum og fejlmeddelelser.
%
Se \url{https://en.wikibooks.org/wiki/LaTeX/Mathematics} for en god oversigt over kommandoer og symboler.


\subsection{Figurer}
\label{sec:figures}

Figurer i \LaTeX{} indsættes som såkaldte \emph{floats} ved brug af \texttt{figure} environmentet.
Et ``flydende'' objekt kan ikke brydes af et sideskift, så en figur vil flyttes, hvis der ikke er plads nok tilbage på siden.
Syntaksen er som følger:
%
\begin{verbatim}
\begin{figure}[placement]
  \centering
  \includegraphics[options]{path/to/image}
  \caption{Billedteksten}
  \label{fig:label}
\end{figure}
\end{verbatim}
%
Argumentet \texttt{placement} kan enten være \texttt{h} (her), \texttt{t} (toppen af siden), \texttt{b} (bunden af siden), eller \texttt{p} (på en særlig side kun for floats).
Kommandoen \verb!\centering! centrerer billedet på siden.
Iblandt de forskellige \texttt{options}, man kan give til \verb!\includegraphics!, så er det vigtigste nok \texttt{width}.
For at få billedet til at fylde halvdelen af siden (margener ikke inkluderet), så brug \verb!width=0.5\textwidth!.
Alle figurer bør have en billedtekst, hvilket sætes med kommandoen \verb!\caption!, og \verb!\label! gør det muligt at referere til den (for eksempel bliver \verb!Figur \ref{fig:me}! til Figur \ref{fig:me}).
For mere info, se \url{https://en.wikibooks.org/wiki/LaTeX/Floats,_Figures_and_Captions}.

% LaTeX template, example figure input file
% ------------------------------------------------------------------------------
% This file contains all the code involved in inputting a single image figure.
% While this code can just as well be inserted directly into the documents with
% the body text, I like to keep my projects as modular as possible.


\begin{figure}[htbp]
  \centering
  \includegraphics[width=0.5\textwidth]{fig/img/me.jpg}
  \caption{A picture of me responding to emails from my students}
  \label{fig:me}
\end{figure}


Figurer behøver ikke nødvendigvist at bestå af billedfiler som JPG, PNG eller PDF, men kan også bestå af \LaTeX{}-kode.
Et vigtigt eksempel på dette er \emph{TiKZ}, som bruges til at tegne vektorgrafik med \LaTeX{}-kommandoer.
Koden til et TiKZ-billede indsættes i environmentet \texttt{tikzpicture}, som indsættes i et \texttt{figure} environment, ligesom billeder:
%
\begin{verbatim}
\begin{figure}[placement]
  \centering
  \begin{tikzpicture}
    % indsæt tikz kode her...
  \end{tikzpicture}
  \caption{Billedtekst}
  \label{fig:label}
\end{figure}
\end{verbatim}
%
Et eksempel på en graf tegnet med TiKZ-kode er vist på Figur \ref{fig:graph}.

% LaTeX template, example tikzpicture input file
% --------------------------------------------------------------------------------------------------
% This file contains all the code involved in inputting a single TikZ figure.  While this code can
% just as well be inserted directly into the documents with the body text, I like to keep my
% projects as modular as possible.


\begin{figure}[htbp]
  \centering

  \begin{tikzpicture}

    % Nodes
    \node[point] at (0,1) (a) [label=left:\(a\)] {};
    \node[point] at (1,2) (b) [label=above:\(b\)] {};
    \node[point] at (1,0) (c) [label=below:\(c\)] {};
    \node[point] at (3,2) (d) [label=above:\(d\)] {};
    \node[point] at (3,0) (e) [label=below:\(e\)] {};
    \node[point] at (4,1) (z) [label=right:\(z\)] {};

    % Edges
    \footnotesize
    \draw (a) -- (b) node[edge] {4};
    \draw (a) -- (c) node[edge] {2};
    \draw (b) -- (c) node[edge] {1};
    \draw (b) -- (d) node[edge] {5};
    \draw (c) -- (d) node[edge] {8};
    \draw (c) -- (e) node[edge] {10};
    \draw (d) -- (e) node[edge] {2};
    \draw (d) -- (z) node[edge] {6};
    \draw (e) -- (z) node[edge] {3};
    
  \end{tikzpicture}

  \caption{Example graph from \citep{rosen}.}
  \label{fig:graph}
\end{figure}


For at lære syntaksen, se \citep{minimaltikz}, tutorial-afsnittene af \citep{tikzmanual}, og \url{https://en.wikibooks.org/wiki/LaTeX/PGF/TikZ}.


\subsection{Tabeller}
\label{sec:tables}

Ligesom med figurer så indsættes tabeller som floats, og dette gøres med \texttt{table} environmentet.
Selve indholdet af tabellen indsættes i et \texttt{tabular} environment, hvor søjlerne separeres af en \verb!&!, og rækkerne separeres af en \verb!\\!:

\begin{verbatim}
  \begin{table}[placement]
    \begin{tabular}{columns}
      % indsæt indholdet af tabellen her...
    \end{tabular}
    \caption{Beskrivelse af tabellen}
    \label{tab:label}
  \end{table}
\end{verbatim}

Argumentet \texttt{placement} er det samme som for figurer.
Argumentet \texttt{columns} er en specifikation af søjlerne og består i sin mest basale form af bogstaverne \texttt{l}, \texttt{c}, og/eller \texttt{r}, muligvis med \verb!|! indsæt i de mellemrum, hvor der skal være lodrette streger.
For eksempel ville \verb!\begin{tabular}{r|cc}! fortælle \LaTeX{}, at tabellen har tre søjler, hvoraf den første er højrejusteret (\texttt{r}) og efterføles af en lodret streg, og de to næste søjler er centrerede uden en streg imellem dem.

For et fuldstændigt eksempel på en tabel, se kildekoden for Tabel \ref{tab:my-table} (kan findes i \verb!fig/tab/my-table.tex!).
For mere informatin og mere avancerede options, se \url{https://en.wikibooks.org/wiki/LaTeX/Tables}.
