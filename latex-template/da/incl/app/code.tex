% LaTeX skabelon, eksempel på appendiks-fil
% ------------------------------------------------------------------------------

\chapter{Python Kode}
\label{app:code}

Hvis I implementerer algoritmer i et specifikt programmeringssprog, vil I muligvis inkludere kildekoden i et appendiks som dette.
Pakken \texttt{listings} gør det muligt at inkludere kildekode med syntax highlighting for mange forskellige sprog.

Med environmentet \texttt{lstlisting} kan man indsætte kode direkte, men for det meste vil I typisk bruge \verb!\lstinputlisting! til at indsætte kode fra en fil, f.eks.
%
\begin{verbatim}
\lstinputlisting[options]{path/to/source_file}
\end{verbatim}
%
Nogle nyttige \texttt{options} for denne kommando er:
%
\begin{description}
\item[caption] sætter caption for koden
\item[label] sætter label for koden (til \verb!\ref!)
\item[language] sætter syntaksen til et bestemt sprog lokalt
\item[style] sætter highlighter stilen lokalt
\item[firstline] springer linjerne før den angivede linje over
\item[lastline] stopper med at læse linjer, når den angivede linje nåes
\end{description}
%
I preamblet kan man også sætte globale options med \verb!\lstset!.
For eksempel, hvis al ens kildekode er skrevet i Python, kan man skrive \verb!\lstset{language=Python}!, og så behøver man ikke bruge sætte \texttt{language} i hver enkel \verb!\lstinputlisting!.

\lstinputlisting[firstline=5,caption=Bubble Sort in Python,label=py:bubble]{py/bubblesort.py}

For mere information, se \url{https://en.wikibooks.org/wiki/LaTeX/Source_Code_Listings}.
