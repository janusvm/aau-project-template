% LaTeX template, example tikzpicture input file
% ------------------------------------------------------------------------------
% This file contains all the code involved in inputting a single TikZ figure.
% While this code can just as well be inserted directly into the documents with
% the body text, I like to keep my projects as modular as possible.


\begin{figure}[htbp]
  \centering

  \begin{tikzpicture}

    % Nodes
    \node[point] at (0,1) (a) [label=left:\(a\)] {};
    \node[point] at (1,2) (b) [label=above:\(b\)] {};
    \node[point] at (1,0) (c) [label=below:\(c\)] {};
    \node[point] at (3,2) (d) [label=above:\(d\)] {};
    \node[point] at (3,0) (e) [label=below:\(e\)] {};
    \node[point] at (4,1) (z) [label=right:\(z\)] {};

    % Edges
    \footnotesize
    \draw (a) -- (b) node[edge] {4};
    \draw (a) -- (c) node[edge] {2};
    \draw (b) -- (c) node[edge] {1};
    \draw (b) -- (d) node[edge] {5};
    \draw (c) -- (d) node[edge] {8};
    \draw (c) -- (e) node[edge] {10};
    \draw (d) -- (e) node[edge] {2};
    \draw (d) -- (z) node[edge] {6};
    \draw (e) -- (z) node[edge] {3};
    
  \end{tikzpicture}

  \caption{Example graph from \citep{rosen}.}
  \label{fig:graph}
\end{figure}
