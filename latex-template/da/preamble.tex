% LaTeX skabelon, preamble-fil
% ------------------------------------------------------------------------------
% Dette er dokumentets preamble, som inkluderer ekstra pakker og definerer
% ekstra kommandoer.


% Basispakker ------------------------------------------------------------------

\usepackage[T1]{fontenc}     % understøttelse til at vise specialtegn i pdf'en
\usepackage[utf8]{inputenc}  % understøttelse til at inputte specialtegn
\usepackage[danish]{babel}   % oversættelse af automatiske ting
\usepackage[square]{natbib}  % sæt referencer til at bruger firkantklammer
\usepackage{hyperref}        % indsæt klik-bare hyperlinks
\usepackage{lastpage}        % referer til sidste side; brugt til at tælle sider


% Matematikpakker --------------------------------------------------------------

\usepackage{mathtools}  % matematik-kommandoer og -environments
\usepackage{amssymb}    % yderligere matematiksymboler
\usepackage{amsthm}     % environments til definitioner, sætninger, osv.
\usepackage{bbm}        % bedre 'blackboard bold' styling


% Figurpakker ------------------------------------------------------------------

\usepackage[dvipsnames]{xcolor}    % til farvet tekst
\usepackage{graphicx}              % håndtering af billedfiler
\usepackage{booktabs}              % bedre tabeller
\usepackage{tikz}                  % til at tegne vektorgrafik
\usepackage{algorithm}             % algoritme-floats
\usepackage[noend]{algpseudocode}  % pseudokode
\usepackage{listings}              % source code


% Layoutpakker mm. -------------------------------------------------------------

\usepackage{microtype}             % bedre mellemrum og font kerning
\usepackage{emptypage}             % fjern sidetal fra tomme sider
\usepackage{fancyhdr}              % headers and footers
\usepackage{titlesec}              % overskrifter
\usepackage[margin=3cm]{geometry}  % justering af sidemargen
\usepackage[toc,page]{appendix}    % bedre appendiks


% Skrifttype -------------------------------------------------------------------
% Skrifttyper er med pdflatex begrænset til dem, der findes pakker til (i
% modsætning til f.eks. xelatex, som lader dig bruge system fonts). Dokumentets
% skrifttype ændres ved blot at importere en pakke til det.
%
% Se
% * http://www.tug.org/pracjourn/2006-1/hartke/hartke.pdf
% * http://www.tug.dk/FontCatalogue/

\usepackage{lmodern}  % bedre standardfont


% Kommandoer mm. ---------------------------------------------------------------

% Danske oversættelser, som ikke fanges af babel-pakken
\renewcommand\appendixpagename{Appendicer}
\renewcommand\appendixtocname{Appendicer}
\renewcommand\lstlistingname{Script}
\renewcommand\lstlistlistingname{Scripts}
\makeatletter
\renewcommand\ALG@name{Algoritme}
\makeatother

% Theorem enviroments
\theoremstyle{plain}
\newtheorem{thm}{Sætning}[chapter]
\newtheorem{lem}[thm]{Lemma}
\newtheorem{prop}[thm]{Proposition}
\newtheorem*{cor}{Korollar}

\theoremstyle{definition}
\newtheorem{defn}[thm]{Definition}
\newtheorem{exmp}[thm]{Eksempel}

% Talmængder
\newcommand{\N}{\mathbb{N}}  % naturlige tal
\newcommand{\Z}{\mathbb{Z}}  % heltal
\newcommand{\Q}{\mathbb{Q}}  % rationelle tal
\newcommand{\R}{\mathbb{R}}  % reelle tal
\newcommand{\C}{\mathbb{C}}  % komplekse tal

% Andre operatorer og begreb
\newcommand{\1}{\mathbbm{1}}     % indikatorfunktion
\newcommand{\bigO}{\mathcal{O}}  % store O

% Headers og footers
\fancyhead{}                      % ryd standardindhold i header
\fancyhead[LE,RO]{\projectgroup}  % venstre på lige sider, højre på ulige
\fancyhead[RE]{\leftmark}         % kapitelnavn
\fancyhead[LO]{\rightmark}        % sektionsnavn
\fancyfoot{}                      % ryd standardindhold i footer
\fancyfoot[CE,CO]{\thepage}       % centrerede sidetal
\pagestyle{fancy}                 % aktiver 'fancy' style

% Titler/overskrifter
% http://mirrors.dotsrc.org/ctan/macros/latex/contrib/titlesec/titlesec.pdf
\titleformat
{\chapter}
[hang]
{\Huge\bfseries}
{\thechapter\enspace\textcolor{BrickRed}{|}\enspace}
{0pt}
{\Huge\bfseries}

% Foruddefinerede TikZ node styles
\tikzstyle{point} = [fill,shape=circle,minimum size=3pt,inner sep=0pt]
\tikzstyle{edge} = [fill=white,midway,inner sep=1pt]

% Syntax highlighting for Python-filer
\lstdefinestyle{custompy}{
  belowcaptionskip=\baselineskip,
  breaklines=true,
  language=Python,
  showstringspaces=false,
  basicstyle=\small\ttfamily,
  keywordstyle=\bfseries\color{green!40!black},
  commentstyle=\itshape\color{purple!40!black},
  identifierstyle=\color{blue},
  stringstyle=\color{orange},
}
\lstset{language=Python,style=custompy,captionpos=b}