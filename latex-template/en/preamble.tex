% LaTeX template, preamble file
% ------------------------------------------------------------------------------
% This is the preamble file for the project, in which package dependencies and
% custom commands should be defined.


% Base packages ----------------------------------------------------------------

\usepackage[T1]{fontenc}     % adds support for printing accented characters
\usepackage[utf8]{inputenc}  % adds support for inputting accented characters
\usepackage[english]{babel}  % set document language
\usepackage[square]{natbib}  % set citation style to use square brackets
\usepackage{hyperref}        % insert clickable hyperlinks
\usepackage{lastpage}        % reference last page (used to count numbered pages)


% Math packages ----------------------------------------------------------------

\usepackage{mathtools}  % math commands and environments
\usepackage{amssymb}    % additional math symbols
\usepackage{amsthm}     % theorem environments
\usepackage{bbm}        % improved 'blackboard bold' styling


% Figure packages --------------------------------------------------------------

\usepackage[dvipsnames]{xcolor}    % font colouring functionality
\usepackage{graphicx}              % handling of images
\usepackage{booktabs}              % improved tables
\usepackage{tikz}                  % drawing vector graphics
\usepackage{algorithm}             % algorithm floats
\usepackage[noend]{algpseudocode}  % pseudocode listings
\usepackage{listings}              % source code listings


% Styling packages -------------------------------------------------------------

\usepackage{microtype}             % improve full justification and font kerning
\usepackage{emptypage}             % suppress page numbering on empty pages
\usepackage{fancyhdr}              % easy customisation of headers and footers
\usepackage{titlesec}              % easy customisation of section titles
\usepackage[margin=3cm]{geometry}  % adjust page margins
\usepackage[toc,page]{appendix}    % improved appendix control


% Font packages ----------------------------------------------------------------
% With pdflatex, fonts are limited to those provided in packages (as opposed to
% xelatex or lualatex, which can use system fonts).  The typeface is changed by
% simply importing the package for it.
%
% See
% * http://www.tug.org/pracjourn/2006-1/hartke/hartke.pdf
% * http://www.tug.dk/FontCatalogue/

\usepackage{lmodern}  % improved default font


% Customisations ---------------------------------------------------------------

% Theorem enviroments
\theoremstyle{plain}
\newtheorem{thm}{Theorem}[chapter]
\newtheorem{lem}[thm]{Lemma}
\newtheorem{prop}[thm]{Proposition}
\newtheorem*{cor}{Corollary}

\theoremstyle{definition}
\newtheorem{defn}[thm]{Definition}
\newtheorem{exmp}[thm]{Example}

% Sets
\newcommand{\N}{\mathbb{N}}  % natural numbers
\newcommand{\Z}{\mathbb{Z}}  % integers
\newcommand{\Q}{\mathbb{Q}}  % rational numbers
\newcommand{\R}{\mathbb{R}}  % real numbers
\newcommand{\C}{\mathbb{C}}  % complex numbers

% Other operators and concepts
\newcommand{\1}{\mathbbm{1}}     % indicator function
\newcommand{\bigO}{\mathcal{O}}  % big O

% Headers and footers
\fancyhead{}                      % clear default header fields
\fancyhead[LE,RO]{\projectgroup}  % left on even, right on odd
\fancyhead[RE]{\leftmark}         % chapter name
\fancyhead[LO]{\rightmark}        % sections name
\fancyfoot{}                      % clear default footer fields
\fancyfoot[CE,CO]{\thepage}       % centered page numbers
\pagestyle{fancy}                 % activate fancy style

% Chapter titles
% http://mirrors.dotsrc.org/ctan/macros/latex/contrib/titlesec/titlesec.pdf
\titleformat
{\chapter}
[hang]
{\Huge\bfseries}
{\thechapter\enspace\textcolor{BrickRed}{|}\enspace}
{0pt}
{\Huge\bfseries}

% Predefined TikZ node styles
\tikzstyle{point} = [fill,shape=circle,minimum size=3pt,inner sep=0pt]
\tikzstyle{edge} = [fill=white,midway,inner sep=1pt]

% Listing style
\lstdefinestyle{custompy}{
  belowcaptionskip=\baselineskip,
  breaklines=true,
  language=Python,
  showstringspaces=false,
  basicstyle=\small\ttfamily,
  keywordstyle=\bfseries\color{green!40!black},
  commentstyle=\itshape\color{purple!40!black},
  identifierstyle=\color{blue},
  stringstyle=\color{orange},
}
\lstset{language=Python,style=custompy,captionpos=b}