% LaTeX template, example appendix file
% --------------------------------------------------------------------------------------------------

\chapter{Python Source Code}
\label{app:code}

If you implement algorithms in specific programming languages, you might want to include the source code in an appendix like this one.
The package \texttt{listings} provides funtionality for including syntax highlighted source code for many different languages.

With the environment \texttt{lstlisting} you can type code directly, but most often you will probably need the \verb!\lstinputlisting! command, which lets you input the code from a source file, i.e.
%
\begin{verbatim}
\lstinputlisting[options]{path/to/source_file}
\end{verbatim}
%
Some notable options for this command are:
%
\begin{description}
\item[caption] Sets the listing caption
\item[label] Sets the listing label
\item[language] Sets the language for the syntax highlighter locally
\item[style] Sets the highlighter style locally
\end{description}
%
In the preamble, you can also set options globally with \verb!\lstset!.
For example, if all your source code is in Python, you might put \verb!\lstset{language=Python}!.

\lstinputlisting[caption=Bubble Sort in Python,label=py:bubble]{py/bubblesort.py}

For more information, see \url{https://en.wikibooks.org/wiki/LaTeX/Source_Code_Listings}.