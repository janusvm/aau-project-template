% LaTeX template, example input file
% --------------------------------------------------------------------------------------------------

\chapter{Second Example}
\label{ch:second-example}

Here is another example, citing \citep{rosen} and \citep{edpenn}.
When you cite a resource, that resource is automatically added to the literature list, and in the PDF file the citations become clickable links pointing to that list.


\section{Custom Environments and Commands}
\label{sec:custom}

While \LaTeX{} provides commands for many different purposes, you will often find yourself defining your own.
For this template, I have included some examples of custom environments and commands in the preamble (\texttt{premable.tex}).

Such commands can save you a lot of typing when working on a long, modular document such as a semester project.
For instance, instead of typing \verb!\mathbb{N}! every time you want the symbol for the set of natural numbers, define a shorter command, like \verb!\N!.
The syntax for defining commands is as follows:
%
\begin{verbatim}
\newcommand{name}[num]{definition}
\end{verbatim}
%
where \texttt{name} is the command name, e.g. \verb!\N!, \texttt{num} is the number of arguments the command takes (omit the square brackets if the command takes no arguments), and \texttt{definition} is the output of the command, e.g. \verb!\mathbb{N}!.


\subsection{Definitions, Theorems, Proofs}
\label{sec:thms}

In a mathematics project, you are going to be including mathematical definitions, propositions, lemmas, theorems, etc.
The \texttt{amsthm} package provides a simple way to define such environments:
%
\begin{verbatim}
\newtheorem{name}{Printed output}[numberby]
\end{verbatim}
%
A few examples are included in the preamble.
See \url{https://en.wikibooks.org/wiki/LaTeX/Theorems} for more information.

\begin{thm}[Example of Theorem]
  \(\sqrt{2}\) is irrational.
\end{thm}
%
\begin{proof}
  Suppose \(\sqrt{2} \in \Q\) and let \(k = \min\{\, N \in \N : N\sqrt{2} \in \N \,\}\).
  But then \(k (\sqrt{2} - 1) = k\sqrt{2} - k \in \N\), and \(k(\sqrt{2} - 1) < k\), which is a contradiction.
  Therefore, \(\sqrt{2} \notin \Q\).
\end{proof}


\subsection{Source and Pseudocode}
\label{sec:code}

WIP...