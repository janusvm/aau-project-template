% LaTeX template, example input file
% --------------------------------------------------------------------------------------------------

\chapter{First Example}
\label{ch:first-example}

This is an example chapter with content.
Aside from \verb!\chapter!, there are several levels available to use for partitioning your body text:
%
\begin{itemize}
\item \texttt{part}
\item \texttt{chapter}
\item \texttt{section}
\item \texttt{subsection}
\item \texttt{subsubsection}
\item \texttt{paragraph}
\item \texttt{subparagraph}
\end{itemize}
%
Each level is a subsection of the above level.
Titles are added automatically to the table of contents.
See more at \url{https://en.wikibooks.org/wiki/LaTeX/Document_Structure#Sectioning_commands}.


% --------------------------------------------------------------------------------------------------
\section{Environments}
\label{sec:sec-within-chap}

In \LaTeX{}, you are going to be using many different kinds of \emph{environments}.
These are scopes denoted with \verb!\begin{...}! and \verb!\end{...}!, enclosing special content such as lists, figures, equations, etc.
Table \ref{tab:my-table} lists some commonly used environments.

\begin{table}[htbp]
  \centering
  \begin{tabular}{ll}
    \toprule
    \textbf{Environment} & \textbf{Function} \\
    \midrule
    \texttt{document} & Document contents \\
    \texttt{table} & Floating table such as this one \\
    \texttt{figure} & Floating figure \\
    \texttt{equation} & Numbered equation \\
    \texttt{align} & Aligned, multiple equations \\
    \texttt{itemize} & Bulleted list \\
    \texttt{enumerate} & Numbered list \\
    \texttt{description} & Descriptive list \\
    \bottomrule
  \end{tabular}
  \caption{Common \LaTeX{} environments and their function}
  \label{tab:my-table}
\end{table}


\subsection{Lists}
\label{sec:lists}

There are three essential list structures: \texttt{itemize}, \texttt{enumerate}, and \texttt{description}.
The \texttt{itemize} variant produces a simple bullet list.
Each item in the list are prepended by the \verb!\item! command.
%
\begin{verbatim}
  \begin{itemize}
    \item First item
    \item Second item
    \item Third item
  \end{itemize}
\end{verbatim}
%
\begin{itemize}
\item First item
\item Second item
\item Third item
\end{itemize}
%
The \texttt{enumerate} variant uses the same syntax for items as \texttt{itemize}, but produces a numbered list.
%
\begin{verbatim}
  \begin{enumerate}
    \item First item
    \item Second item
    \item Third item
  \end{enumerate}
\end{verbatim}
%
\begin{enumerate}
\item First item
\item Second item
\item Third item
\end{enumerate}
%
Finally, the \texttt{description} list in which \verb!\item! is given an item name as an optional argument, and the contents of the line is a description of that item.
This produces a list where the item names are typeset in bold followed by their descriptions as normal text.
%
\begin{verbatim}
  \begin{description}
    \item[First item] Description of first item
    \item[Second item] Description of second item
    \item[Third item] Description of third item
  \end{description}
\end{verbatim}
%
\begin{description}
\item[First item] Description of first item
\item[Second item] Description of second item
\item[Third item] Description of third item
\end{description}


\subsection{Equations}
\label{sec:equations}

One of the main reasons why people use \LaTeX{} is the beautiful math typesetting.
There are several different math environments to suit your needs, and most come in a numbered and unnumbered variants.
For example, the code
%
\begin{verbatim}
\begin{equation}
  \label{eq:1}
  e^{i\pi} - 1 = 0
\end{equation}
\end{verbatim}
%
produces the ouput
%
\begin{equation}
  \label{eq:1}
  e^{i\pi} + 1 = 0 ,
\end{equation}
%
and since it was given a label, it can be referenced with the command \verb!\eqref{eq:1}!, which produces a clickable reference in parentheses, \eqref{eq:1}.
If instead of \texttt{equation} you put \texttt{equation*}, the equation does not get a number.
Equivalently, you can use \verb!\[...\]!, so the code \verb!\[ e^{i\pi} + 1 = 0 \]! produces
\[ e^{i\pi} + 1 = 0 .\]

If you need multiple, aligned equations, e.g. for step-by-step calculations, use the \texttt{align} environment, which aligns the contents at \texttt{\&} characters.
For example,
%
\begin{verbatim}
\begin{align*}
  (x + y)^{2} &= x^{2} + xy + yx + y^{2} \\
              &= x^{2} + y^{2} + 2xy
\end{align*}
\end{verbatim}
%
produces
%
\begin{align*}
  (x + y)^{2} &= x^{2} + xy + yx + y^{2} \\
              &= x^{2} + y^{2} + 2xy .
\end{align*}
%
The double backslash denotes a line break.
Note the asterisk; like with \texttt{equation}, \texttt{align} has both a numbered and unnumbered version.
The numbered version has a seperate number for each line.

For rendering inline math, e.g. \(\cos^{2}\theta + \sin^{2}\theta = 1\), use \verb!\(...\)!.
Alternatively, you can also use \verb!$...$!, but \verb!\(...\)! has improved spacing and error messages.

See \url{https://en.wikibooks.org/wiki/LaTeX/Mathematics} for a good reference of symbols and commands.


\subsection{Floats}
\label{sec:floats}

Figures in \LaTeX{} are input as so-called \emph{floats} using the \texttt{figure} environment.
A floating object cannot be broken over a page, so the figure will be repositioned depending on the available space on the page.
The syntax is as follows:
%
\begin{verbatim}
\begin{figure}[placement]
  \centering
  \includegraphics[options]{path/to/image}
  \caption{The figure caption}
  \label{fig:label}
\end{figure}
\end{verbatim}
%
The optional argument \texttt{placement} can be either of \texttt{h} (here), \texttt{t} (top of page), \texttt{b} (bottom of page), or \texttt{p} (put on special page with only floats).
The \verb!\centering! command is there to center the image.
Among the options available for \verb!\includegraphics!, the most important one for you will probably be \texttt{width}.
To make the image take up half the page (within margins), use \verb!width=0.5\textwidth!.
All figures should have a caption, which is set with the \verb!\caption! command, and the \verb!\label! lets us reference it (for example, \verb!Figure \ref{fig:me}! becomes Figure \ref{fig:me}).

% LaTeX template, example figure input file
% ------------------------------------------------------------------------------
% This file contains all the code involved in inputting a single image figure.
% While this code can just as well be inserted directly into the documents with
% the body text, I like to keep my projects as modular as possible.


\begin{figure}[htbp]
  \centering
  \includegraphics[width=0.5\textwidth]{fig/img/me.jpg}
  \caption{A picture of me responding to emails from my students}
  \label{fig:me}
\end{figure}


For more info, see \url{https://en.wikibooks.org/wiki/LaTeX/Floats,_Figures_and_Captions}.
